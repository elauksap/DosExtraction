\hypertarget{index_intro}{}\section{Introduction}\label{index_intro}
This program allows to extract the Density of States (Do\-S), assessed by means of capacitance-\/voltage measurements, in an organic semiconductor device. Simulated values are fitted to experimental data. \par
All source and header files are written in C++11 language. \par
The software is intended to be used on a Unix-\/like operating system.\hypertarget{index_dependancies}{}\section{Dependancies}\label{index_dependancies}
The program requires the following software to be installed on your system\-:

\begin{DoxyItemize}
\item \href{http://www.cmake.org}{\tt C\-Make} (version 2.\-8 or above), a cross-\/platform configuration tool; \item \href{http://www.gnu.org/software/make}{\tt Make} (version 3.\-8.\-1 or above), a tool used to build executables; \item \label{index_GCC}%
\hypertarget{index_GCC}{}%
\href{http://www.gnu.org/software/gcc}{\tt G\-C\-C} (version 4.\-8 or above), the G\-N\-U Compiler Collection; \par
 \par
\item \label{index_Eigen}%
\hypertarget{index_Eigen}{}%
\href{http://eigen.tuxfamily.org}{\tt Eigen} (version 3.\-2 or above), to handle with matrices, vectors and linear algebra; \item \label{index_Gnuplot}%
\hypertarget{index_Gnuplot}{}%
\href{http://www.gnuplot.info}{\tt Gnuplot} (version 4.\-6.\-4 or above), a graphical utility to generate plots (the package {\bfseries gnuplot-\/x11}, a terminal for X servers, is also required for the interactive interface); \item \href{http://www.boost.org}{\tt Boost} (version 1.\-50 or above), a set of libraries used by \hyperlink{index_gnuplot-iostream}{gnuplot-\/iostream}.\end{DoxyItemize}
It also uses the following libraries, shipped in the {\itshape include/} folder\-:

\begin{DoxyItemize}
\item \href{http://getpot.sourceforge.net}{\tt Get\-Pot} (version 1.\-1.\-18), to parse command-\/line and configuration files; \item \label{index_gnuplot-iostream}%
\hypertarget{index_gnuplot-iostream}{}%
\href{http://www.stahlke.org/dan/gnuplot-iostream}{\tt gnuplot-\/iostream} (version 2), a C++ interface for \hyperlink{index_Gnuplot}{Gnuplot}.\end{DoxyItemize}
Parallel computing capabilities are provided through the \href{http://openmp.org}{\tt Open\-M\-P} library, shipped together with \hyperlink{index_GCC}{G\-C\-C}.\hypertarget{index_install_sec}{}\section{Compile}\label{index_install_sec}
In order to generate the executable, first open the {\itshape C\-Make\-Lists.\-txt} file (in the top-\/level folder) and, if necessary, edit it to your needs.

Then create a build directory and move into it\-:


\begin{DoxyCode}
$ mkdir build
$ cd build
\end{DoxyCode}


Now you're ready to configure your system\-:


\begin{DoxyCode}
$ cmake ..
\end{DoxyCode}


\begin{DoxyNote}{Note}
or, should you want the compiler to produce also debug symbols\-:


\begin{DoxyCode}
$ cmake -DCMAKE\_BUILD\_TYPE=Debug ..
\end{DoxyCode}

\end{DoxyNote}
and finally\-:


\begin{DoxyCode}
$ make
\end{DoxyCode}


will build the {\itshape simulate\-\_\-dos} executable and the {\itshape dosextraction} shared library under the {\itshape bin/} and {\itshape lib/} directories (or the ones specified in {\itshape C\-Make\-Lists.\-txt}) respectively.

If you wish to install\-:

\begin{DoxyItemize}
\item the executable, into {\itshape /usr/local/bin/}; \item the shared library, into {\itshape /usr/local/lib/}; \item the header files, into {\itshape /usr/local/include/dosextraction/};\end{DoxyItemize}
just type\-:


\begin{DoxyCode}
\textcolor{preprocessor}{# make install}
\end{DoxyCode}


while\-:


\begin{DoxyCode}
\textcolor{preprocessor}{# make uninstall}
\end{DoxyCode}


will remove them. \par
 If \href{http://www.doxygen.org}{\tt Doxygen} (version 3.\-8.\-6 or above) and \href{http://www.graphviz.org}{\tt Graph\-Viz} are found, the following command will generate the present documentation under the {\itshape doc/} folder (or the one specified in {\itshape C\-Make\-Lists.\-txt})\-:


\begin{DoxyCode}
$ make doc
\end{DoxyCode}
\hypertarget{index_configure}{}\section{Set up the configurations}\label{index_configure}
\begin{DoxyNote}{Note}
The default configuration directory is {\itshape config/}.
\end{DoxyNote}
Before you can run the executable, you have to set up the configuration file (default\-: {\itshape config.\-pot}). Within it you can find a list of parameters, each of which is commented out to explain what modifying it will entail. \par
Particularly, the variables {\itshape input\-\_\-params} and {\itshape input\-\_\-experim} can be set, i.\-e. the filenames where to find input fitting parameters and experimental data respectively. It's recommended (but not compulsory) to put these files in the same directory as the configuration file (otherwise you can specify a relative or absolute path to them).

\begin{DoxyWarning}{Warning}
The program never checks that the input values are numeric but will always cast them to floating point numbers, then please pay attention while setting up the variable {\itshape skip\-Headers}.
\end{DoxyWarning}
You can create multiple configuration files, each with different parameter values\-: the one you aim to use can be specified in the command-\/line before running.\hypertarget{index_run}{}\section{Run!}\label{index_run}
To run by using the default configuration filename ({\itshape config.\-pot}) simply move into the directory where the executable is located in and type\-:


\begin{DoxyCode}
$ ./simulate\_dos
\end{DoxyCode}


By default, the configuration file is searched in {\itshape }../config. \par
 To specify a different configuration file previously saved in the configuration directory\-:


\begin{DoxyCode}
$ ./simulate\_dos -f configuration\_filename
\end{DoxyCode}


or\-:


\begin{DoxyCode}
$ ./simulate\_dos --file configuration\_filename
\end{DoxyCode}


The variable {\itshape configuration\-\_\-filename} should {\bfseries not} contain the path.

\begin{DoxyWarning}{Warning}
Furthermore, if you run the program from a different folder or if you chose a different configuration directory, you have also to manually specify the path to the configuration directory (either absolute or relative to the current directory) by using\-:


\begin{DoxyCode}
$ ./simulate\_dos -d configuration\_directory
\end{DoxyCode}


or\-:


\begin{DoxyCode}
$ ./simulate\_dos --directory configuration\_directory
\end{DoxyCode}

\end{DoxyWarning}
Once complete, the results of the simulation(s) will be saved in the output directory (relative to the current folder) specified in the configuration file (default\-: {\itshape output/}). \par
\hyperlink{index_Gnuplot}{Gnuplot} scripts are saved too for later re-\/use under the {\itshape gnuplot/} subdirectory; you can run them through\-:


\begin{DoxyCode}
$ gnuplot name\_of\_the\_script
\end{DoxyCode}
 