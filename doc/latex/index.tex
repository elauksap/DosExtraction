\hypertarget{index_intro}{}\section{Introduction}\label{index_intro}
This program allows to extract the Density of States, assessed by mean capacitance-\/voltage measurements, in an organic semiconductor device. Simulated values are fitted to experimental data. \par
Source and header files are written in C++11 language. \par
The software is intended to be used on a Unix-\/like operating system.\hypertarget{index_dependancies}{}\section{Dependancies}\label{index_dependancies}
The program requires the following libraries to be installed on your system\-:

\begin{DoxyItemize}
\item \label{index_Eigen}%
\hypertarget{index_Eigen}{}%
\href{http://eigen.tuxfamily.org}{\tt {\bfseries Eigen}}, to handle with matrices, vectors and linear algebra; \item \label{index_Gnuplot}%
\hypertarget{index_Gnuplot}{}%
\href{http://www.gnuplot.info}{\tt {\bfseries Gnuplot}}, a graphical utility to generate plots; \item \href{http://www.boost.org}{\tt {\bfseries Boost}}, a C++ library used by the \hyperlink{index_Gnuplot}{Gnuplot} interface to C++; \item \href{http://openmp.org}{\tt {\bfseries Open\-M\-P}}, for parallel computing (recommended but not compulsory).\end{DoxyItemize}
It also uses the following packages, provided in the {\itshape include/} folder\-: \begin{DoxyItemize}
\item \href{http://getpot.sourceforge.net}{\tt {\bfseries Get\-Pot}}, to parse command-\/line and configuration files; \item \href{http://www.stahlke.org/dan/gnuplot-iostream}{\tt {\bfseries gnuplot-\/iostream}}, the C++ interface for \hyperlink{index_Gnuplot}{Gnuplot}.\end{DoxyItemize}
\hypertarget{index_install_sec}{}\section{Compile}\label{index_install_sec}
In order to compile a test executable, simply execute one of these commands in a terminal pointing to the root directory of this package\-:


\begin{DoxyCode}
$ make
\end{DoxyCode}


or, if you want the compiler to produce also debugging informations\-:


\begin{DoxyCode}
$ make debug
\end{DoxyCode}


You can specify the name of the test to be compiled (without extension, for example\-: {\itshape simulate\-\_\-dos}) by passing the variable {\bfseries N\-A\-M\-E} through command-\/line\-:


\begin{DoxyCode}
$ make NAME=test\_filename
\end{DoxyCode}


The compiler will generate the {\itshape test\-\_\-filename} executable under the {\itshape bin/} directory. \par
Repeat these instructions for each test you want to compile.\hypertarget{index_configure}{}\section{Set up the configurations}\label{index_configure}
\begin{DoxyNote}{Note}
The default configuration directory is {\itshape config/}.
\end{DoxyNote}
Before you can run an executable, you have to set up the configuration file (default\-: {\itshape config.\-pot}). Within it you can find a list of parameters, each of which is commented out to explain what modifying it will entail. \par
Particularly, the variables {\itshape input\-\_\-params} and {\itshape input\-\_\-experim} can be set, i.\-e. the filenames where to find input fitting parameters and experimental data respectively. It's recommended (but not compulsory) to put these files in the same directory as the configuration file (otherwise you can specify a relative or absolute path to them). \par
\par
You can create multiple configuration files, each with different parameter values\-: the one you aim to use can be specified in the command-\/line before running.\hypertarget{index_run}{}\section{Run!}\label{index_run}
Executables are placed under the {\itshape bin/} directory. \par
\par
To run by using the default configuration filename ({\itshape config.\-pot}) simply move to the {\itshape bin/} directory and execute\-:


\begin{DoxyCode}
$ ./test\_filename
\end{DoxyCode}


To specify a different configuration file previously saved in the configuration directory\-:


\begin{DoxyCode}
$ ./test\_filename -f configuration\_filename
\end{DoxyCode}


or\-:


\begin{DoxyCode}
$ ./test\_filename --file configuration\_filename
\end{DoxyCode}


The variable {\itshape configuration\-\_\-filename} should {\bfseries not} contain the path.

\begin{DoxyWarning}{Warning}
Furthermore, if you run the program from a different folder than {\itshape bin/} or if you chose a different configuration directory, you have to manually specify the {\bfseries full} path to the configuration directory (either absolute or relative to the current directory) by using\-:


\begin{DoxyCode}
$ ./test\_filename -d configuration\_directory
\end{DoxyCode}


or\-:


\begin{DoxyCode}
$ ./test\_filename --directory configuration\_directory
\end{DoxyCode}

\end{DoxyWarning}
Once complete, you can find the results of the simulation(s) in the output directory specified in the configuration file (default\-: {\itshape output/}) under {\itshape bin/}. 